%% LyX 2.1.4 created this file.  For more info, see http://www.lyx.org/.
%% Do not edit unless you really know what you are doing.
\documentclass[english]{article}
\usepackage[T1]{fontenc}
\usepackage[latin9]{inputenc}
\usepackage{verbatim}
\usepackage{graphicx}

\makeatletter
%%%%%%%%%%%%%%%%%%%%%%%%%%%%%% Textclass specific LaTeX commands.
\newenvironment{lyxlist}[1]
{\begin{list}{}
{\settowidth{\labelwidth}{#1}
 \setlength{\leftmargin}{\labelwidth}
 \addtolength{\leftmargin}{\labelsep}
 \renewcommand{\makelabel}[1]{##1\hfil}}}
{\end{list}}

\makeatother

\usepackage{babel}
\begin{document}
ARTIQ 2-Tone Waveform Gateware

\paragraph{Overview}

Gateware, testing and ARTIQ integration is needed for a high data
rate, two-tone, interpolating, scalable, high-speed arbitrary waveform
generator. The starting point for the following tasks is the PDQ hardware
designed by NIST. The hardware is discussed in the online literature. 
\begin{quote}
Arbitrary waveform generator for quantum information processing with
trapped ions; R. Bowler, U. Warring, J. W. Britton, B. C. Sawyer and
J. Amini; Rev. Sci. Instrum. 84, 033108 (2013) \end{quote}
\begin{verbatim}
	tf.boulder.nist.gov/general/pdf/2668.pdf
\end{verbatim}
Improved gateware for the PDQ hardware due to Robert Jordens is called
PDQ2. Documentation on PDQ2 is publicly available. 
\begin{verbatim}
	github.com/m-labs/pdq2
	pdq2.readthedocs.org/en/latest
\end{verbatim}
Terminology and techniques referenced in this Task are introduced
by these online resources.


\paragraph{Demonstration Hardware}

Target hardware for waveform generator is the following. 
\begin{enumerate}
\item DAC is Analog Devices AD9154-FMC-EBZ prototype board
\item FPGA, DRAM, transceivers, etc. are those found on Xilinx KCU105 prototyping
board
\item a 125 MHz differential LVTTL clock shall be supplied by University 
\end{enumerate}

\paragraph{Waveform Compression}

Waveform data is to be represented in the compressed format defined
by Robert Jordens for the PDQ2. What follows is from Robert's readthedocs
writeup.

The method of compression is a polynomial basis spline (B-spline).
The data consists of a sequence of knots. Each knot is described by
a duration $\Delta t$ and spline coefficients $u_{n}$ up to order
$k$. If the knot is evaluated starting at time $t_{0}$, the output
$u(t)$ for $t\in[t_{0},t_{0}+\Delta t]$ is
\[
u(t)=\sum_{n=0}^{k}\frac{u_{n}}{n!}(t-t_{0})^{n}=u_{0}+u_{1}(t-t_{0})+\frac{u_{2}}{2}(t-t_{0})^{2}+...
\]
A sequence of such knots describes a spline waveform. From one discrete
time $i$ to the next $i+1$each accumulator $v_{n,i}$ is incremented
by the value of the next higher order accumulator:
\[
v_{n,i+1}=v_{n,i}+v_{n+1,i}
\]
For a cubic spline the mapping between accumulators' initial values
$v_{n,0}$ and the polynomial derivatives or spline coefficients $u_{n}$
can be done off-line and must take into consideration the finite time
step size $\tau$. The data for each knot is described by the integer
duration $T=\Delta t/\tau$ and the initial values $v_{n,0}$. This
representation allows both very fast transient high bandwidth waveforms
and slow but smooth large duty cycle waveforms to be described efficiently.

Alternate spline representations may be selected by mutual agreement
of the University and vendor. 


\paragraph{Parameterization}

Splines are parameterized as follows.
\begin{itemize}
\item $u$ and $a$ are 16-bit cubic spline interpolators ({*})
\item $p$ are 16-bit constants 
\item $f$ are 48-bit linear interpolators ({*}) 
\item $b$ are $1$ or $0$
\item $ep(t)$, $ef(t)$, $ea(t)$, $eu(t)$ are 16-bit constants
\item $f_{DATA}\geq1$ ~GHz 
\item ({*}) interpolators are updated at $f_{DATA}/8$, 16-bit knot duration
\item timestamps are 64-bit
\end{itemize}
Alternate spine parameterizations may be selected by mutual agreement
of the University and vendor. 


\paragraph{Waveform Specification}

\begin{comment}
From 7/29 conversation with R Jordens...

Every channel 

Re( (a1{*}exp(i{*}(f1{*}t+p1)) + a2{*}exp(i{*}(f2{*}t+p2))) {*} exp(i{*}(f0{*}t+p0))
) + optionally Im( signal-from-the-adjacent-channel ) 

Each channel does: Re( (a1{*}exp(i{*}(f1{*}t+p1)) + a2{*}exp(i{*}(f2{*}t+p2)))
{*} exp(i{*}(f0{*}t+p0)) ) + optionally Im( signal-from-the-adjacent-channel
)

with all splines (a,f,p) being sampled at 156 MHz, the 1 and 2 oscillators
being sampled at 156 MHz, and the 0 term at 8x156 MHz. modulo a lot
of important cross checks that i still have to do, this will get us
to \textasciitilde{}62\% LUT usage on a kc705. it includes some ARTIQ
stuff, CPUs etc but not the actual JESD204B stuff, the transcievers,
and a bunch of other things. oh. 8 channels, 16 bits
\end{comment}


A waveform consisting of the sum of two independently modulated tones
is required. Call this waveform W2T.
\[
Z=(a_{1}e^{i(f_{1}t+p_{1})}+a_{2}e^{i(f_{2}t+p_{2})})e^{i(f_{0}t+p_{0})}
\]


\[
out_{\mbox{W2T}}=u_{0}+b_{R}Re\{Z\}+b_{I}Im\{Z\}
\]

\begin{itemize}
\item spines $a,p$ and linear interpolators $f$ are updated at $f_{DATA}/8$ 
\item oscillators $(f_{1},p_{1})$ and $(f_{2},p_{2})$ are sampled at $f_{DATA}/8$
\item oscillator $(f_{0},p_{0})$ is sampled at $f_{DATA}=f_{DAC}/2$ (2X
interpolation)
\item let $pfaub\equiv\{$ $p_{0}$, $p_{1}$, $p_{2}$, $f_{0}$, $f_{1}$,
$f_{2}$, $a_{1}$, $a_{2}$,$u_{0}$, $b_{R}$, $b_{I}\}$
\item gateware must support parallel generation of W2T on at least 8 independent
channels
\item if an external clock is selected in ARTIQ configuration, waveform
output shall be phase synchronous with this externally supplied clock
\item implementation shall support external modulation 

\begin{itemize}
\item where $ep_{\mbox{DRTIO}}(t)$, $ef_{\mbox{DRTIO}}(t)$, $ea_{\mbox{DRTIO}}(t)$
and $eu_{\mbox{DRTIO}}(t)$ 

\begin{itemize}
\item are $t$-aligned with interpolator updates
\item have default values at $t=0$ of $ep_{\mbox{DRTIO}}=0$, $ef_{\mbox{DRTIO}}=0$,
$ea_{\mbox{DRTIO}}=1$ and $eu_{\mbox{DRTIO}}=0$ 
\item may be individually set by DRTIO using an API exposed in ARTIQ Python 
\end{itemize}
\item where $ep_{\mbox{PID}}(t)$, $ef_{\mbox{PID}}(t)$, $ea_{\mbox{PID}}(t)$
and $eu_{\mbox{PID}}(t)$ 

\begin{itemize}
\item are $t$-aligned with $f_{DAC}/npid$ where $npid$$\leq2000$
\item have default values at $t=0$ of $ep_{\mbox{PID}}=0$, $ef_{\mbox{PID}}=0$,
$ea_{\mbox{PID}}=1$ and $eu_{\mbox{PID}}=0$ 
\item anticipate future PID feedback
\end{itemize}
\item where for $i\in[1,2]$ 
\[
\begin{array}{l}
p_{i}=p_{i}+ep_{i,\mbox{DRTIO}}(t)+ep_{i,\mbox{PID}}(t)\mbox{ where \ensuremath{i\in[0,1,2]}}\\
f_{i}=f_{i}+ef_{i,\mbox{DRTIO}}(t)+ef_{i,\mbox{PID}}(t)\mbox{ where \ensuremath{i\in[0,1,2]}}\\
a_{i}=a_{i}\cdot ea_{i,\mbox{DRTIO}}(t)\cdot ea_{i,\mbox{PID}}(t)\mbox{ where \ensuremath{i\in[1,2]}}\\
u_{0}=u_{0}+eu_{0,\mbox{DRTIO}}(t)+eu_{0,\mbox{PID}}(t)
\end{array}
\]


\begin{itemize}
\item the following are independent for each DAC channel: $ep_{\mbox{DRTIO}}(t)$,
$ef_{\mbox{DRTIO}}(t)$, $ea_{\mbox{DRTIO}}(t)$, $eu_{\mbox{DRTIO}}(t)$,
$ep_{\mbox{PID}}(t)$, $ef_{\mbox{PID}}(t)$, $ea_{\mbox{PID}}(t)$
and $eu_{\mbox{PID}}(t)$ 
\item $|ea_{DRTIO}|\leq1$ and $|ea_{PID}|\leq1$
\item it remains to be determined how to prevent/trap overflow of $u_{0}$
due to $eu_{0,\mbox{DRTIO}}(t)+eu_{0,\mbox{PID}}(t)$
\end{itemize}
\end{itemize}
\item support for storing and indexing pre-computed spline waveforms 

\begin{itemize}
\item include a spline waveform selection API exposed in ARTIQ Python 
\item be implemented foreseeing future implementations where spline waveforms
are 

\begin{itemize}
\item sourced from core device memory with a sustained aggregated spline
knot rate $\geq2$ MHz
\item sourced from memory on Sayma
\item computed dynamically on core device CPU
\end{itemize}
\end{itemize}
\item parallel changes to spline waveforms affecting multiple DAC channels
must be synchronous 
\item minimum spline knot duration is $\leq8$ ns 

\begin{itemize}
\item updating all channels all splines in parallel, maximum burst length
for spline waveforms is $\geq128$ knots per spline
\end{itemize}
\end{itemize}

\paragraph{Interface }
\begin{itemize}
\item an API for ARTIQ Python shall include

\begin{itemize}
\item sequencing of indexed pre-computed spline waveforms
\item setting of any of $pfaub$ in series or in parallel for all DAC channels
\item setting of $ep_{\mbox{DRTIO}}(t)$, $ef_{\mbox{DRTIO}}(t)$, $ea_{\mbox{DRTIO}}(t)$
and $eu_{\mbox{DRTIO}}(t)$ by DRTIO
\item setting arbitrary registers on AD9154 
\end{itemize}
\item utility functions shall be made available to users for processing
spline waveforms

\begin{itemize}
\item given a periodically sampled waveform (vector of amplitude, time)
routines shall

\begin{itemize}
\item generate a spline waveform with a fixed knot duration 
\item generate a spline waveform with specified knot count and variable
knot duration 
\end{itemize}
\item given user-supplied spline waveform routines shall

\begin{itemize}
\item generate a periodically sampled waveform (vector of amplitude, time)
with user specified period
\item check for accumulator overflow during time evolution of the spline
waveform
\end{itemize}
\end{itemize}
\end{itemize}

\paragraph{test cases}

Demonstration ARTIQ Python programs will provide examples of the following
test cases.
\begin{lyxlist}{00.00.0000}
\item [{tc-a}] Simultaneous generation of two-tone waveforms parameterized
by W2T on 8 DAC channels where $f_{1}=f_{0}+\Delta$ and $f_{2}=f_{0}-\Delta$
where $f_{0}=200$~MHz and $\Delta=[0,50]$~MHz. For demonstration
on KCU105, data must be generated for all 8 DAC channels but only
4 will be routed to AD9154-FMC-EBZ via HPC FMC. 
\item [{phys-b}] TODO for LogiQ: using physics nomenclature define a prototypical
experiment involving 32 ions.
\item [{tc-b}] $ep_{\mbox{DRTIO}}(t)$ is intended to permit efficient
update of DAC phase after each gate. TODO 
\item [{tc-sync-a}] Support for agile, synchronous operations that permit
reproduction of behavior PHASE\_MODE\_CONTINUOUS, PHASE\_MODE\_ABSOLUTE
and PHASE\_MODE\_TRACKING as currently supported by ARTIQ for AD9914.
\item [{tc-sync-b}] tc-sync-a shall produce identically phased results
between power cycles of KCU105.
\end{lyxlist}



\end{document}
