\documentclass[english]{article}
\usepackage{verbatim}
\usepackage{graphicx}
\usepackage{hyperref}
\usepackage{babel}
\begin{document}

\section*{ARTIQ-SAYMA\footnote{Smart ArbitrarY waveform ModulAtor}}

\paragraph{Overview}

Specification and design concept for a high data rate, multi-tone, interpolating, scalable, smart, high-speed arbitrary waveform generator.

\paragraph{Spline parametrization}

\emph{This is inherited from the \href{http://pdq2.readthedocs.io/en/latest/}{pdq2 documentation}}

The method of compression is a polynomial basis spline (B-spline).
The data consists of a sequence of knots. Each knot is described by
a duration $\Delta t$ and spline coefficients $u_{n}$ up to order
$k$. If the knot is evaluated starting at time $t_{0}$, the output
$u(t)$ for $t\in[t_{0},t_{0}+\Delta t]$ is
$$
u(t)=\sum_{n=0}^{k}\frac{u_{n}}{n!}(t-t_{0})^{n}=u_{0}+u_{1}(t-t_{0})+\frac{u_{2}}{2}(t-t_{0})^{2}+...
$$
A sequence of such knots describes a spline waveform. From one discrete
time $i$ to the next $i+1$ each accumulator $v_{n,i}$ is incremented
by the value of the next higher order accumulator:
$$
v_{n,i+1}=v_{n,i}+v_{n+1,i}
$$
For a cubic spline the mapping between accumulators' initial values
$v_{n,0}$ and the polynomial derivatives or spline coefficients $u_{n}$
can be done off-line and must take into consideration the finite time
step size $\tau$. The data for each knot is described by the integer
duration $T=\Delta t/\tau$ and the initial values $v_{n,0}$. This
representation allows both transient large-bandwidth waveforms
and slow but smooth large duty cycle waveforms to be described very efficiently.

\paragraph{Waveform parametrization}

The gateware will support at least 8 independent channels.

Each channel emits waveforms of the general parametrization:
$$
	z=\left( a_1e^{i(f_1 t+p_1)} + a_2e^{i(f_2 t+p_2)}\right)e^{i(f_0t+p_0)}
$$
$$
	o=u+b_0\mathrm{Re}(z)+b_1\mathrm{Im}(z^\prime)
$$

\begin{itemize}
	\item $o$ is the (real valued) output of a channel
	\item $z$ is the complex-valued output of the ``generator'' associated with each channel
	\item $z^\prime$ is the complex-valued output from the generator of each channel's ``buddy'' channel.
		Two adjaccent channels form a buddy pair.
		This enables seamless usage of the complex data path features in DACs, complex (IQ) analog modulation, and yields ``four-tone'' support on IQ channels for free.
	\item $u$ and $a$ are 16-bit cubic (third order) spline interpolators
	\item $p$ are 16-bit constant (zeroth order) spline interpolators
	\item $f$ are 48-bit linear (first order) interpolators
	\item $b$ are switches ($1$ or $0$)
\end{itemize}

\paragraph{Datapath details}

\begin{itemize}
	\item $f_\mathrm{DATA}\geq1\,\mathrm{GHz}$. Exact clock speed is TBD and depends on simultaneously meeting hardware constraints and an integer relationship with the RTIO clock and physics/noise requirements.
	\item Oscillator $f_0,p_0$ is sampled at $f_{DATA}$
	\item Interpolators are updated and interpolate at $f_\mathrm{DATA}/k$ with $k$ typically 4 or 8 and $f_\mathrm{DATA}/k \geq 125\,\mathrm{MHz}$
	\item Oscillators $f_{1,2},p_{1,2}$ are sampled at $f_{DATA}/k$
	\item All amplitude summing junctions shall implement saturating summation to prevent wrap-around.
	\item All amplitude summing junctions shall implement configurable and guaranteed gateware low-high limiters.
	\item All amplitude summing junctions shall register saturation events.
	\item To up-sample the data from the $f_1$,$f_2$ oscillators by $k$ before passing it into the $f_0$ oscillator, a CIC filter of order TBD shall be implemented for anti-aliasing. CIC filters are linear phase.
	\item To implement further anti-aliasing, a symmetric (thus linear phase) FIR filter with TBD taps (FPGA DSP resource limits) shall be implemented after the CIC filter.
	\item All spline interpolators and the total channel output shall be monitored by the ARTIQ channel monitoring infrastructure.
	\item All spline interpolators shall support ARTIQ injection/override.
\end{itemize}

\paragraph{Clocking and synchronization}

\begin{itemize}
	\item Timestamps for spline knot scheduling are at least 62 bit wide.
	\item Spline knots have 16-bit dynamic range in time.
	\item In order to support slower sweeps with sparser spline knots, the dynamic range of the spline coefficients can be extended using time stretcher.
		It decelerates the spline evolution/interpolation rate by a factor of $2^E$.
	\item Waveform output shall be with deterministic latency with respect to the RTIO clock:
		\begin{itemize}
			\item across channels on the same card (to within DAC
				chip specification)
			\item across cards in the same rack (to within DAC chip
				and intra-rack DRTIO clock sycnchronization)
			\item across racks controlled by the same core device
				(to within DAC chip and DRTIO clock synchronization)
		\end{itemize}
	\item Each card can be clocked by an internal DAC clock derived from the RTIO clock or by an external DAC clock.
	\item When an external DAC clock is used, the waveform synchronization is ensured to within one DAC clock cycle (or the limit of the DAC chip whichever is higher) but below that depends on the phase of the external DAC clock.
	\item All spline knot interpolators can be updated independently (and also simultaneously) of each other.
	\item All spline interpolator latencies from the internal ``RTIO clock reference plane'' to the DAC output are matched and deterministic.
		Channel and board latencies are matched and deterministic (see above).
	\item Minimum spline knot duration is $k/f_\mathrm{DATA}$. 
\end{itemize}

\paragraph{Phase update modes}

The phase accumulator of the DDS cores can be updated in multiple different modes during a phase and/or frequency update.

\begin{itemize}
	\item relative phase update: $q^\prime(t) = q(t^\prime) + (p^\prime - p) + (t - t^\prime) f^\prime$
	\item absolute phase update: $q^\prime(t) = p^\prime + (t - t^\prime) f^\prime$
	\item phase coherent update: $q^\prime(t) = p^\prime + (t - T) f^\prime$, where
	\item $q$/$q^\prime$: old/new phase accumulator
	\item $p$/$p^\prime$: old/new phase offset
	\item $f^\prime$: new frequency
	\item $t^\prime$: timestamp of setting new $p$,$f$
	\item $T$: ``origin'' timestamp: beginning of experiment, boot of device, or arbitrary
	\item $t$: running time
\end{itemize}

Relative phase updates are called ``continuous phase mode'' and coherent updates are called ``tracking phase mode'' by some.
Phase coherent updates can be mapped (in software/runtime) to absolute phase updates by transforming $p^\prime \longrightarrow p^\prime + (t^\prime - T) f^\prime$.
Since phase coherent updates require large multiplications is is questionable whether they can and should be implemented in gateware.

It is questionable whether phase coherent updates should or even can be supported for sweeping $p$/$f$. They can be supported for the modulation inputs (see below).

\paragraph{Modulation by RTIO}

To each spline interpolator (any of the nine $f,p,a,u$ in the waveform
parametrization) a modulation (summarized as $e_\mathrm{RTIO}$) by a separate RTIO channel can be applied.


\begin{itemize}
	\item The modulation is an additive offset for frequency and phase ($f,p$) and a multiplicative offset for amplitudes ($u,a$).
	\item The modulation is times like any other (non-interpolating) RTIO event, i.e. $\leq 8$\,ns time resolution and has the same value resolution as the spline interpolator it modulates.
	\item Default values are 0 for frequency and phase modulation ($f,p$) and 1 for amplitude modulation ($u,a$).
	\item Modulation is normalized to full scale.
\end{itemize}

\paragraph{Modulation by local DSP}
In addition to RTIO modulation $e_\mathrm{RTIO}$ there is ``local DSP'' modulation
input to each spline interpolator.

\begin{itemize}
	\item Same specifications and semantics as the RTIO modulation.
\end{itemize}

\paragraph{Local DSP}
A fully reconfigurable local DSP fabric with multiple IIR filters shall be
included. The DSP switchyard supports servoing applications of various types.
\begin{itemize}
	\item See \href{https://github.com/jordens/redpid}{redpid} for a rough
		feature set.
\end{itemize}

\paragraph{Runtime and kernel interface}
\begin{itemize}
	\item Spline knot sequences can be generated off-line and embedded in ARTIQ experiments.
	\item Spline knot sequences can be generated at compile time.
	\item Spline knot sequences can be embedded into ARTIQ experiments and emitted to from the core device to the DRTIO channels during the experiments.
	\item Spline knot sequences can be computed dynamically on core device.
	\item Instead of emitting them directly to the DRTIO channel, spline knot sequences can be emitted into a named DMA context which stores the RTIO events in memory (either on the core device or right at the DRTIO channel in the card's DRAM) for later recall.
	\item Stored, named DMA segments can be replayed by name.
	\item Given enough slack to transmit DRTIO events and fill the channel FIFOs (from core device or from any DMA source), all boards, all channels, all splines can burst $\geq128$ knots each at $\geq125$\,MHz (BRAM FIFO limited). This is independent of whether the events are computed dynamically, off-line, embedded, reside in core device DRAM or remote DRAM.
	\item When sourcing waveforms from core device memory, the sustained aggregated spline knot rate across all interpolators is $\geq2$\,MHz.
	\item Sourcing from remote DRTIO DMA the spline knot rate per board (aggregated over all channels and all interpolators on that board) is TBD MHz sustained for TBD knots (DRAM limited).
	\item Supports setting $e_\mathrm{DRTIO}$ using standard DRTIO events.
	\item Supports configuring the DAC through RTIO-SPI
	\item Utility functions shall be made available to users for processing spline waveforms (scaling in value and time, resampling).
	\item Given a periodically sampled waveform (vector of values) routines shall
	\begin{itemize}
		\item generate a spline waveform with a fixed knot duration 
		\item generate a spline waveform with specified knot count and variable knot duration 
		\item generate a spline waveform with minimal knot count and specified RMS error
	\end{itemize}
	\item given user-supplied spline waveform routines shall
	\begin{itemize}
		\item generate a periodically sampled waveform (vector of values) with user specified resolution
		\item determine validity (in-range)
	\end{itemize}

\end{itemize}

\paragraph{Test Cases}

ARTIQ Python programs demonstrating the following will be provided.

\begin{enumerate}
	\item Simultaneous generation of two-tone waveforms on 8 DAC channels where $f_{1}=f_{0}+\Delta$ and $f_{2}=f_{0}-\Delta$ where $f_{0}=200$~MHz and $\Delta=[0,50]$~MHz.
	\item Playing a spline knot sequence demonstrating each spline interpolator in
		turn.
	\item Replaying a 128 knot two-tone amplitude sequence from remote DMA.
	\item Phase/frequency/amplitude shifting that sequence using $e_\mathrm{DRTIO}$.
	\item Demonstrate relative and absolute phase mode.
	\item Demonstrate deterministic channel alignment to one DAC clock cycle.
	\item Demonstrate external and internal clocking.
\end{enumerate}

\end{document}
